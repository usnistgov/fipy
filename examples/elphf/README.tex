The following examples exhibit various parts of a model to study
electrochemical interfaces.  In a pair of papers, Guyer, Boettinger, Warren
and McFadden \cite{ElPhFI,ElPhFII} have shown that an electrochemical
interface can be modeled by an equation for the phase field \( \xi \)

\[
    \frac{\partial \xi}{\partial t}
    = 
    M_{\xi}\kappa_{\xi}\nabla^2 \xi
    - 
    M_{\xi}\sum_{j=1}^{n} C_j \left[
	p'(\xi) \Delta\mu_j^\circ
	+ g'(\xi) W_j
    \right]
    +
    M_{\xi}\frac{\epsilon'(\xi)}{2}\left(\nabla\phi\right)^2
\]

a set of diffusion equations for the concentrations \( C_j \), for \(
j = 2,\ldots, n-1 \), of the substitutional elements

\begin{align*}
    \frac{\partial C_j}{\partial t}
    &= D_{j}\nabla^2 C_j \\
    & \qquad + 
	D_{j}\nabla\cdot 
	\frac{C_j}{1 - \sum_{\substack{k=2\\ k \neq j}}^{n-1} C_k}
	\left\{
	    \sum_{\substack{i=2\\ i \neq j}}^{n-1} \nabla C_i
	    + 
	    C_n \left[
		p'(\xi) \Delta\mu_{jn}^{\circ}
		+ g'(\xi) W_{jn}
	    \right] \nabla\xi
	    +
	    C_n z_{jn} \nabla \phi
	\right\}
\end{align*}

a diffusion equation for the concentration \( C_{\text{e}^{-}} \) of
electrons

\[
    \frac{\partial C_{\text{e}^{-}}}{\partial t}
    = D_{\text{e}^{-}}\nabla^2 C_{\text{e}^{-}} \\
    + D_{\text{e}^{-}}\nabla\cdot 
	C_{\text{e}^{-}}
	\left\{
	    \left[
		p'(\xi) \Delta\mu_{\text{e}^{-}}^{\circ}
		+ g'(\xi) W_{\text{e}^{-}}
	    \right] \nabla\xi
	    +
	    z_{\text{e}^{-}} \nabla \phi
	\right\}
\]

and Poisson's equation for the electrostatic potential \( \phi \)

\[ 
    \nabla\cdot\left(\epsilon\nabla\phi\right) 
    +
    \rho
    = 0
\]

\( M_\xi \) is the phase field mobility, \( \kappa_\xi \) is the phase
field gradient energy coefficient, \( p'(\xi) =
30\xi^2\left(1-\xi\right)^2 \), and \( g'(\xi) =
2\xi\left(1-\xi\right)\left(1-2\xi\right) \).  For a given species \(
j \), \( \Delta\mu_j^{\circ} \) is the standard chemical potential
difference between the electrode and electrolyte for a pure material,
\( W_j \) is the magnitude of the energy barrier in the double-well
free energy function, \( z_j \) is the valence, and \( D_{j} \) is the
self diffusivity.  \( \Delta\mu_{jn}^{\circ} \), \( W_{jn} \), and \(
z_{jn} \) are the differences of the respective quantities \(
\Delta\mu_{j}^{\circ} \), \( W_{j} \), and \( z_{j} \) between
substitutional species \( j \) and the solvent species \( n \).  The
total charge is denoted by \( \rho \equiv \sum_{j=1}^n z_j C_j \).

The module \verb+fipy.models.elphf+ has been developed to solve this
coupled set of equations.  Although unresolved stiffnesses make the
full solution intractable in \FiPy{}, we can demonstrate the use of
various parts of the \verb+elphf+ module.
