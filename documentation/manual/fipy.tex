%% -*-TeX-*-
 % ###################################################################
 %  FiPy - a finite volume PDE solver in Python
 % 
 %  FILE: "fipy.tex"
 %                                    created: 4/1/04 {2:58:37 PM} 
 %                                last update: 7/6/04 {12:13:53 PM} 
 %  Author: Jonathan Guyer
 %  E-mail: guyer@nist.gov
 %  Author: Daniel Wheeler
 %  E-mail: daniel.wheeler@nist.gov
 %    mail: NIST
 %     www: http://ctcms.nist.gov
 %  
 % ========================================================================
 % This document was prepared at the National Institute of Standards
 % and Technology by employees of the Federal Government in the course
 % of their official duties.  Pursuant to title 17 Section 105 of the
 % United States Code this document is not subject to copyright
 % protection and is in the public domain.  fipy.tex
 % is an experimental work.  NIST assumes no responsibility whatsoever
 % for its use by other parties, and makes no guarantees, expressed
 % or implied, about its quality, reliability, or any other characteristic.
 % We would appreciate acknowledgement if the document is used.
 % 
 % This document can be redistributed and/or modified freely
 % provided that any derivative works bear some notice that they are
 % derived from it, and any modified versions bear some notice that
 % they have been modified.
 % ========================================================================
 % See the file "license.terms" for information on usage and  redistribution of
 % this file, and for a DISCLAIMER OF ALL WARRANTIES.
 %  
 % ###################################################################
 %%

\documentclass[letterpaper]{book}

\usepackage[text={7in,9.333in}]{geometry}

\usepackage{crop}

\usepackage{alltt, parskip, boxedminipage} % fancyheadings,  
\usepackage{multirow, longtable, makeidx, amssymb} 
\usepackage{amsmath} 

\usepackage{minitoc}
\setcounter{tocdepth}{1}
\setcounter{parttocdepth}{1}

\usepackage[ascii]{inputenc}

%\usepackage{multirow, longtable, tocbibind, amssymb} % makeidx, 
% \usepackage{fullpage}
%\usepackage[headings]{fullpage}
\makeindex
\usepackage[usenames]{color}
\definecolor{darkblue}{rgb}{0,0.05,0.35}
\definecolor{redish}{rgb}{0.894,0.122,0.122}
\definecolor{bluish}{rgb}{0.216,0.188,0.533}
% \usepackage[dvips, pagebackref, pdftitle={FiPy}, pdfcreator={epydoc 2.1}, bookmarks=true, bookmarksopen=false, pdfpagemode=UseOutlines, colorlinks=true, linkcolor=black, anchorcolor=black, citecolor=black, filecolor=black, menucolor=black, pagecolor=black, urlcolor=darkblue]{hyperref}
\usepackage[urlcolor=blue,linkcolor=blue,bookmarksopen, pdftex, pagebackref, pdftitle={FiPy}, pdfcreator={epydoc 2.1}, bookmarks=true, bookmarksopen=false, pdfpagemode=UseOutlines,colorlinks=true,plainpages=false,pdfpagelabels]{hyperref}
\usepackage{graphicx}
% \usepackage{memhfixc}
% \usepackage{nameref}

\graphicspath{{../figures/}}

% for reStructuredText
\usepackage{shortvrb}
\usepackage{tabularx}
\setlength{\extrarowheight}{2pt}
\newlength{\admonitionwidth}
\setlength{\admonitionwidth}{0.9\textwidth}
\newlength{\docinfowidth}
\setlength{\docinfowidth}{0.9\textwidth}
\newlength{\locallinewidth}
\newcommand{\optionlistlabel}[1]{\bf #1 \hfill}
\newenvironment{optionlist}[1]
{\begin{list}{}
  {\setlength{\labelwidth}{#1}
   \setlength{\rightmargin}{1cm}
   \setlength{\leftmargin}{\rightmargin}
   \addtolength{\leftmargin}{\labelwidth}
   \addtolength{\leftmargin}{\labelsep}
   \renewcommand{\makelabel}{\optionlistlabel}}
}{\end{list}}
% begin: floats for footnotes tweaking.
\setlength{\floatsep}{0.5em}
\setlength{\textfloatsep}{\fill}
\addtolength{\textfloatsep}{3em}
\renewcommand{\textfraction}{0.5}
\renewcommand{\topfraction}{0.5}
\renewcommand{\bottomfraction}{0.5}
\setcounter{totalnumber}{50}
\setcounter{topnumber}{50}
\setcounter{bottomnumber}{50}
% end floats for footnotes
% some commands, that could be overwritten in the style file.
\newcommand{\rubric}[1]{\subsection*{~\hfill {\it #1} \hfill ~}}
\newcommand{\titlereference}[1]{\textsl{#1}}


\newcommand{\logo}{\rotatebox{4}{\textcolor{redish}{\( \varphi \)}}\kern-.70em\raisebox{-.15em}{\textcolor{bluish}{\( \pi\)}}}
% \newcommand{\logoToo}{\raisebox{-.15em}{\textcolor{bluish}{\(\pi\)}}\kern-.64em\rotatebox{4}{\textcolor{redish}{\( \varphi \)}}}

\newcommand{\FiPy}{\textsf{FiPy}}

% \includeonly{installation}

\makeatletter
\renewcommand*\l@section{\@dottedtocline{1}{1.5em}{3.3em}}
\renewcommand*\l@subsection{\@dottedtocline{2}{3.8em}{4.2em}}
\renewcommand*\l@subsubsection{\@dottedtocline{3}{7.0em}{5.1em}}
\renewcommand*\l@paragraph{\@dottedtocline{4}{10em}{6em}}
\renewcommand*\l@subparagraph{\@dottedtocline{5}{12em}{7em}}

\renewcommand\tableofcontents{%
    \if@twocolumn
      \@restonecoltrue\onecolumn
    \else
      \@restonecolfalse
    \fi
    \chapter*{\contentsname\pdfbookmark[-1]{Contents}{contents}
        \@mkboth{%
           \MakeUppercase\contentsname}{\MakeUppercase\contentsname}}%
    \@starttoc{toc}%
    \if@restonecol\twocolumn\fi
    }

\renewcommand\maketitle{\begin{titlepage}%
\let\footnotesize\small
\let\footnoterule\relax
\let \footnote \thanks
\null\vfil
\vskip 20\p@
\begin{flushright}%
  {\Huge \@title \par}%
  \vskip 3em%
  {\large
   \lineskip .75em%
    \begin{tabular}[t]{r@{}}%
      \@author
    \end{tabular}\par}%
    \vskip 1.5em%
  {\large \@date \par}%       % Set date in \large size.
\end{flushright}\par
\@thanks
\vfil\null
\end{titlepage}%
\setcounter{footnote}{0}%
\global\let\thanks\relax
\global\let\maketitle\relax
\global\let\@thanks\@empty
\global\let\@author\@empty
\global\let\@date\@empty
\global\let\@title\@empty
\global\let\title\relax
\global\let\author\relax
\global\let\date\relax
\global\let\and\relax
}

\makeatother

\usepackage{layout}

\begin{document}

\doparttoc

\crop[frame,axes]

\frontmatter

% \layout

% \settypeblocksize{9in}{7in}{*}
% \setlrmargins{*}{*}{*}
% \setulmargins{*}{*}{*}

% \settrimmedsize{8.5in}{11in}{*}       % pi/2
% % \settypeblocksize{40\onelineskip}{*}{0.61803}  % golden ratio
% \settypeblocksize{33\onelineskip}{*}{0.61803}  % golden ratio
% \setlength{\trimtop}{0pt}
% \setlength{\trimedge}{\stockwidth}
% \addtolength{\trimedge}{-\paperwidth}
% \addtolength{\trimedge}{-1.5in}
% \setlrmargins{*}{*}{2}
% % \setlrmargins{*}{1in}{*}
% \setulmargins{5\onelineskip}{*}{*}
% \setheadfoot{3\onelineskip}{3\onelineskip}
% \setheaderspaces{\onelineskip}{*}{*}
% \setmarginnotes{17pt}{65pt}{\onelineskip}

% \checkandfixthelayout

% \fixpdflayout

% \setlength{\parindent}{0ex}
\setlength{\fboxrule}{2\fboxrule}
\newlength{\BCL} % base class length, for base trees.

% \pagestyle{Ruled}
\renewcommand{\sectionmark}[1]{\markboth{#1}{}}
\renewcommand{\subsectionmark}[1]{\markright{#1}}

\newenvironment{Ventry}[1]%
  {\begin{list}{}{%
    \renewcommand{\makelabel}[1]{\texttt{##1:}\hfil}%
    \settowidth{\labelwidth}{\texttt{#1:}}%
    \setlength{\leftmargin}{\labelsep}%
    \addtolength{\leftmargin}{\labelwidth}}}%
  {\end{list}}

\newsavebox{\NISTbox}
\sbox{\NISTbox}{\includegraphics[trim=5 2 5 5,scale=1.5]{NIST_right2line}}
%\newlength{\photoheight}
%\settoheight{\photoheight}{\usebox{\photobox}}
%\newlength{\photowidth}
%\settowidth{\photowidth}{\usebox{\photobox}}

\makeatletter
\newcommand*{\ps@mytitlepage}{%
  \let\@oddhead\@empty
  \renewcommand*{\@oddfoot}{%
  \hfill\parbox[b][0pt]{0pt}{\makebox[0pt][r]{\usebox{\NISTbox}}}}}%\vspace*{-0.5in}}
\makeatother

\makeatletter  
\begin{titlepage}%
\thispagestyle{mytitlepage}
\let\footnotesize\small
\let\footnoterule\relax
\let \footnote \thanks
% \begin{flushleft}
%     \scalebox{10}{\logo}
% \end{flushleft}
% \null\vfil
% \vskip 20\p@
\begin{flushright}%
  \scalebox{10}{\logo}\par
%   \vspace*{\fill}
  \vskip 3em%
  {\Huge FiPy \\
    \huge A Finite Volume PDE Solver Using Python \par}%
  \vskip 3em%
  {\large
   \lineskip .75em%
    \begin{tabular}[t]{r@{}}%
        Daniel Wheeler \\
        Jonathan E. Guyer \\ 
        James A. Warren \\
        \textit{Metallurgy Division} \\
        \textit{Materials Science and Engineering Laboratory}
    \end{tabular}\par}%
    \vskip 1.5em%
  {\large \@date \par}%       % Set date in \large size.
% \vfil\null
  \vspace*{\fill}%
%   \raisebox{-5cm}{\fbox{\includegraphics[hiresbb,scale=1.5]{NIST_right2line}}}%
\end{flushright}%
\end{titlepage}%
\makeatother

% \title{
% \scalebox{5}{\logo} \\[2ex]
% \Huge FiPy \\
% A Finite Volume PDE Solver Using Python \\
% }
% 
% \author{Daniel Wheeler, Jonathan E. Guyer \& James A. Warren \\
% Metallurgy Division, Materials Science and Engineering Laboratory\\
% National Institute of Standards and Technology}
% 
% \maketitle

\tableofcontents

\mainmatter

\part{Introduction}

\parttoc

\chapter{Installation}

\inputencoding{latin1}

\input installation

\inputencoding{ascii}

% \include{intro}

\chapter{Numerical Methods}

This chapter describes the numerical methods used to solve equations
in the \FiPy{} programming environment. FiPy{} uses the Finite Volume
Method (FVM) to solve coupled sets of Partial Differential Equations
(PDEs).

Essentially, the FVM consists of dividing the solution domain into
discrete finite volumes over which the state variables are
approximated with linear or higher order interpolations. The
interpolations are satisfied with simple approximations to the
derivatives in each term of the equation. This process is known as
discretization. The (FVM) is a popular discretization technique
employed to solve coupled PDEs used in many application areas
(e.g. Fluid Dynamics).

\input numerical/equation
\input numerical/discret
\input numerical/scheme




\part{Examples}

\parttoc

\chapter{Diffusion Examples}

\input{examples/latex/examples.diffusion.steadyState.mesh1D.input-module}
\input{examples/latex/examples.diffusion.steadyState.mesh20x20.input-module}
\input{examples/latex/examples.diffusion.steadyState.mesh50x50.input-module}
\input{examples/latex/examples.diffusion.explicit.mesh10.input-module}
\input{examples/latex/examples.diffusion.explicit.mesh50.input-module}
\input{examples/latex/examples.diffusion.variable.mesh2x1.input-module}
\input{examples/latex/examples.diffusion.variable.mesh10x1.input-module}
\input{examples/latex/examples.diffusion.variable.mesh50x1.input-module}
\input{examples/latex/examples.diffusion.nthOrder.input2ndOrder1D-module}
\input{examples/latex/examples.diffusion.nthOrder.input4thOrder1D-module}

\chapter{Convection Examples}

\include{examples/latex/examples.convection.exponential1D.input-module}
\include{examples/latex/examples.convection.exponential1DBack.input-module}
\include{examples/latex/examples.convection.exponential1DSource.input-module}
\include{examples/latex/examples.convection.exponential2D.input-module}
\include{examples/latex/examples.convection.powerLaw1D.input-module}

\chapter{Phase Field Examples}

\include{examples/latex/examples.phase.anisotropy.input-module}
\include{examples/latex/examples.phase.impingement.mesh40x1.input-module}
\include{examples/latex/examples.phase.impingement.mesh20x20.input-module}
\include{examples/latex/examples.phase.impingement.restart.input-module}
\include{examples/latex/examples.phase.missOrientation.circle.input-module}
\include{examples/latex/examples.phase.missOrientation.mesh1D.input-module}
\include{examples/latex/examples.phase.missOrientation.modCircle.input-module}
\include{examples/latex/examples.phase.symmetry.input-module}

\chapter{Level Set Examples}

\include{examples/latex/examples.levelSet.distanceFunction.oneD.input-module}
\include{examples/latex/examples.levelSet.distanceFunction.square.input-module}
\include{examples/latex/examples.levelSet.distanceFunction.circle.input-module}
\include{examples/latex/examples.levelSet.distanceFunction.interior.input-module}
\include{examples/latex/examples.levelSet.advection.mesh1D.input-module}
\include{examples/latex/examples.levelSet.advection.circle.input-module}







% \include{examples}
% 
% \include{theory}
% 
% \include{code}
 
\appendix

\part{APIs}

\parttoc

\input api

\backmatter

%%%%%%%%%%%%%%%%%%%%%%%%%%%%%%%%%%%%%%%%%%%%%%%%%%%%%%%%%%%%%%%%%%%%%%%%%%%
%%                                 Index                                 %%
%%%%%%%%%%%%%%%%%%%%%%%%%%%%%%%%%%%%%%%%%%%%%%%%%%%%%%%%%%%%%%%%%%%%%%%%%%%

\pdfbookmark[-1]{Index}{index}
\printindex


\end{document}
