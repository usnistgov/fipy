\section{General Conservation Equation}

A general conservation equation, solved using \FiPy{}, can include any
combination of the following terms,
\begin{equation}                        
  \underbrace{
  \frac{\partial (\rho \phi)}{\partial t}
  }_{\text{transient}}
=
\underbrace{
  \vphantom{\frac{\partial (\rho \phi)}{\partial t}}
  \nabla \cdot \left( \vec{u} \phi \right)
}_{\text{convection}}
+
\underbrace{
  \vphantom{\frac{\partial (\rho \phi)}{\partial t}}
  \nabla \cdot \left( \Gamma_1 \nabla \phi \right) 
}_{\text{diffusion}}
+
\underbrace{
  \vphantom{\frac{\partial (\rho \phi)}{\partial t}}
  \left[ \nabla \cdot \left( \Gamma_i \nabla \right) \right]^n \phi
}_{\text{n\textit{th} order}}
+
\underbrace{
  \vphantom{\frac{\partial (\rho \phi)}{\partial t}}
  S_{\phi}
}_{\text{source}}
\label{eqn:num:gen}
\end{equation}
where $\rho$, $\vec{u}$ and $\Gamma_i$ represent coefficients in the
transient, convection and diffusion terms respectively. These
coefficients can be arbitrary functions of any parameters or variables
in the system. The variable $\phi$, represents the unknown quantity in
the equation. The n\textit{th} order term can represent any higher
order diffusion-type term depending on the exponent $n$. For example,
the n\textit{th} order term can represent a diffusion term when the
exponent $n = 1$ or a Cahn-Hilliard term when $n = 2$. A Cahn-Hilliard
term has the following form,
\begin{equation}
  \nabla \cdot \left( \Gamma_1 \nabla \left[ 
  \nabla \cdot \left( \Gamma_2 \nabla \phi \right) \right] \right)
  \label{eqn:cahn-hilliard}
\end{equation}
Higher order terms are also possible for $n > 2$.



