\section{General Conservation Equation}
\label{sec:GeneralConservationEquation}

The equations that model the evolution of physical, chemical and
biological systems often have a remarkably universal form.  Indeed,
PDEs have proven necessary to model complex physical systems and
processes that involve variations in both space and time.  In general,
given a variable of interest \( \phi \) such as species concentration,
pH, or temperature, there exists an evolution equation of the form
\begin{equation}
  \frac{\partial \phi}{\partial t} = H(\phi, \lambda_i)
  \label{eqn:general-equation}
\end{equation}
where \( H \) is a function of \(\phi\), other state variables
\(\lambda_i\), and higher order derivatives of all of these variables.
Examples of such systems are wide ranging, but include problems that
exhibit a combination of diffusing and reacting species, as well as
such diverse problems as determination of the electric potential in
heart tissue, of fluid flow, stress evolution, and even the
Schr\"odinger equation.

A general conservation equation, solved using \FiPy{}, can include any
combination of the following terms,
\begin{equation}                        
  \underbrace{
  \frac{\partial (\rho \phi)}{\partial t}
  }_{\text{transient}}
=
\underbrace{
  \vphantom{\frac{\partial (\rho \phi)}{\partial t}}
  \nabla \cdot \left( \vec{u} \phi \right)
}_{\text{convection}}
+
\underbrace{
  \vphantom{\frac{\partial (\rho \phi)}{\partial t}}
  \nabla \cdot \left( \Gamma_1 \nabla \phi \right) 
}_{\text{diffusion}}
+
\underbrace{
  \vphantom{\frac{\partial (\rho \phi)}{\partial t}}
  \left[ \nabla \cdot \left( \Gamma_i \nabla \right) \right]^n \phi
}_{\text{$n$\textsuperscript{th} order}}
+
\underbrace{
  \vphantom{\frac{\partial (\rho \phi)}{\partial t}}
  S_{\phi}
}_{\text{source}}
\label{eqn:num:gen}
\end{equation}
where $\rho$, $\vec{u}$ and $\Gamma_i$ represent coefficients in the
transient, convection and diffusion terms respectively.  These
coefficients can be arbitrary functions of any parameters or variables
in the system.  The variable $\phi$, represents the unknown quantity
in the equation.  The $n$\textsuperscript{th} order term can represent
any higher order diffusion-type term, where the order is given by the
exponent $n$.  For example, the $n$\textsuperscript{th} order term can
represent a diffusion term when the exponent $n = 1$ or a
Cahn-Hilliard term when $n = 2$.  A Cahn-Hilliard term has the
form~\cite{CahnHilliardI,CahnHilliardII,CahnHilliardIII},
\begin{equation}
  \nabla \cdot \left( \Gamma_1 \nabla \left[ 
  \nabla \cdot \left( \Gamma_2 \nabla \phi \right) \right] \right)
  \label{eqn:cahn-hilliard}
\end{equation}
Of course, higher order terms ($n > 2$) are also possible.



